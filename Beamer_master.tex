\documentclass{beamer}
\usetheme{metropolis} % Requires metropolis theme installed

\usepackage{amsmath}
\usepackage[T1]{fontenc}
\usepackage{bookmark}
\usepackage{graphicx}
\usepackage{hyperref}
\usepackage{booktabs}
\usepackage{xcolor}
\usepackage{colortbl}

\definecolor{Azul}{RGB}{29,66,191}
\definecolor{myGreen}{RGB}{12, 170, 50} 
\definecolor{Amarelo}{RGB}{242,203,46}
\definecolor{lightblue}{RGB}{173, 216, 230}
\definecolor{lblack}{gray}{0.85} 

\setbeamercolor{frametitle}{bg=Azul, fg=white}
\setbeamercolor{title}{fg=Azul} 
\setbeamercolor{structure}{fg=Azul} 
\setbeamercolor{progress bar}{fg=Azul} 
\setbeamercolor{title separator}{fg=black}
\setbeamercolor{alerted text}{fg=Azul}

\begin{document}

\begin{frame}
    \frametitle{\Large{Research Design:} DUBOIS Lucas (s216283)}
    \begin{itemize}
        \item \textbf{Research Question:} \textit{How do dividend tax cuts impact top-income inequality?}
        \item \textbf{Objective:} This paper analyzes the effect of Brazil’s 1995 dividend tax cut and explores whether the observed rise in top-income inequality was a consequence of this policy change.
        \item \textbf{Approach:} Quantitative analysis using historical data from the World Inequality Database (\href{https://wid.world/country/brazil/}{WID}) and national sources such as IBGE.
        \item \textbf{Methodology:} A combination of econometric models will be employed to estimate the causal effect of the 1995 tax reform. These include panel regressions with fixed effects, a difference-in-differences (DiD) approach comparing Brazil to similar countries, and a synthetic control method (SCM). The models will incorporate relevant controls (e.g., GDP growth, inflation, other tax policies) to ensure robustness and validity.
    \end{itemize}
\end{frame}

\begin{frame}
    \frametitle{\LARGE Literature Review:}
    \begin{itemize}
        \item Becker, G. S. (1968). Crime and punishment: An economic approach. Journal of Political Economy, 76(2), 169–217.
        \\\vspace{0.1cm}
        \hspace{0.5cm}\textit{key insights}: Crime from a rational standpoint: humans commit crimes when the expected benefits exceed the expected costs.
    \end{itemize}
\end{frame}

\end{document}
